\documentclass{article}
\usepackage[margin=0.8in]{geometry}
\usepackage{amsmath}
\usepackage{amssymb}
\usepackage{float}
\usepackage{graphicx}
\usepackage{bookmark}
\newcommand{\tr}{\text{tr}}
\newcommand{\xbold}{\mathbf{x}}
\newcommand{\wbold}{\mathbf{w}}
\newcommand{\ind}{\mathbb{I}}

\title{Solutions to Assignment 2 : CS6510 - Applied Machine Learning}
\author{Vishwak S\\
\texttt{CS15BTECH11043}}
\date{}

\begin{document}
\maketitle

\section*{Question 1}
\subsection*{Part a}
\begin{flushleft}
In the expression for probabilities:
\begin{equation}
P(Y = K | X = \xbold) = \frac{1}{1 + \displaystyle \sum_{t=1}^{K-1} \exp(\wbold_{t}^{T}\xbold)} = \frac{\exp(\wbold_{K}^{T}\xbold)}{\displaystyle \sum_{t=1}^{K} \exp(\wbold_{t}^{T}\xbold)} \text{ with } \wbold_{K} = \mathbf{0}
\end{equation}
The above change results in a better looking and easy solution.

Denote \(W = \{\wbold_{1}, \wbold_{2}, \ldots, \wbold_{K}\}\). Hence writing the likelihood \(+\) L2-regularization, the objective becomes:
\begin{equation}
L'(W) = \displaystyle \prod_{j=1}^{D}P(Y = y_{j} | X = \xbold_{j} ; W) - \frac{\lambda}{2}\sum_{t=1}^{K}||\wbold_{t}||^{2}
\end{equation}

Hence the log-likelihood becomes:
\begin{equation}
L(W) = \displaystyle \sum_{j=1}^{D} \ln(P(Y = y_{j} | X = \xbold_{j} ; W) - \frac{\lambda}{2}\sum_{t=1}^{K}||\wbold_{t}||^{2}
\end{equation}

Note that for \(Y = y_{j}\), only the weight vector \(\wbold_{y_{j}}\) will be utilized. Hence \(P(Y = y_{j} | X = \xbold_{j}) = \displaystyle \frac{\exp(\wbold_{y_{j}}^{T}\xbold_{j})}{\displaystyle \sum_{t=1}^{K} \exp(\wbold_{t}^{T}\xbold_{j})}\). Using this, we get \(L\) to be:
\begin{equation}
L(W) = \displaystyle \sum_{j=1}^{D} \wbold_{y_{j}}^{T}\xbold_{j} - \ln\left(\sum_{t=1}^{K}\exp(\wbold_{t}^{T}\xbold_{j})\right) - \frac{\lambda}{2}\sum_{t=1}^{K} ||\wbold_{t}||^{2}
\end{equation}

The negative sign in the regularizer is because of the maximizing objective, and minimizing the regularization term is the same as maximizing the negative of it.
\end{flushleft}

\subsection*{Part b}
\begin{flushleft}
We need to find \(\nabla_{\wbold_{i}} L(W)\).
\begin{gather}
\nabla_{\wbold_{i}} L(W) = \sum_{j=1}^{D} \left[\left(\ind_{y_{j} = i \neq K}\right) \xbold_{j} - \frac{1}{\displaystyle \sum_{t=1}^{K}\exp(\wbold_{t}^{T}\xbold_{j})} \exp(\wbold_{i}^{T}\xbold_{j}) \left(\ind_{i \neq K}\right) \xbold_{j}\right] - \lambda \wbold_{i} \left(\ind_{i \neq K}\right) \\
\nabla_{\wbold_{i}} L(W) = \ind_{i \neq K} \left(\sum_{j=1}^{D} \left[ \left(\ind_{y_{j} = i}\right) \xbold_{j} - \frac{1}{\displaystyle \sum_{t=1}^{K}\exp(\wbold_{t}^{T}\xbold_{j})} \exp(\wbold_{i}^{T}\xbold_{j}) \xbold_{j}\right] - \lambda \wbold_{i} \right)
\end{gather}

Here \(\ind\) denotes the indicator function. The use of the indicator function \(\ind_{i \neq K}\) is that, the actual objective function is independent of \(\wbold_{K}\). Hence the gradient of \(L(W)\) w.r.t. \(\wbold_{K}\) will always be \(\mathbf{0}\), and this ensures that. The use of \(\ind_{y_{j} = i}\) comes from the utility of \(w_{i}\) in the expression for \(P(y = y_{j} | X = \xbold_{j})\), where we mentioned that \(w_{y_{j}}\) is only used.
\end{flushleft}

\subsection*{Part c}
Below is the update equation for gradient ascent with learning rate \(\eta\):
\begin{gather}
\wbold_{k} := \wbold_{k} + \eta \nabla_{\wbold_{k} = \mathbf{0}}L(W = \mathbf{0}) \\
\wbold_{k} := \mathbf{0} + \eta \left[\ind_{k \neq K} \displaystyle \sum_{j=1}^{D} \left(\ind_{y_{j} = k}\right)\xbold_{j} - \frac{\xbold_{j}}{K}\right] = \eta \ind_{k \neq K} \displaystyle \sum_{j=1}^{D} \left(\left(\ind_{y_{j} = k}\right)\xbold_{j} - \frac{\xbold_{j}}{K}\right)
\end{gather}

\subsection*{Part d}
Yes, an appropriate choice of \(\lambda\) can result in the \(-L(W)\) being strongly convex, consequently \(L(W)\) will become strongly concave and we will be able to reach a global maximum.
\subsection*{Part e}
% TODO: Check internet

\section*{Question 2}
% TODO: Change both of the answers
\subsection*{Part a}
The dendrogram is below, circles represent clusters.
\begin{figure}[H]
\centering
\includegraphics[width=0.3\linewidth]{./images/2a.png}
\end{figure}

\subsection*{Part b}
The dendrogram is below, circles represent clusters.
\begin{figure}[H]
\centering
\includegraphics[width=0.3\linewidth]{./images/2b.png}
\end{figure}

\subsection*{Part c}
Changing the lowest distance i.e., \(d(x_1, x_2) = 0.12 \rightarrow 0.09\) and the highest distance i.e., \(d(x_3, x_6) = 0.93 \rightarrow 0.96\) will not affect the answers.

\subsection*{Part d}
\begin{flushleft}
Assume spherical covariances i.e., the covariance matrix of the K mixtures is \(\Sigma_{k} = \Sigma = \alpha I \hspace{4mm} \forall k \in \{1, \ldots, K\}\). Using Bayes' theorem:
\begin{equation}
p(z_{k} = 1 | \xbold) = \frac{p(z_{k} = 1)p(\xbold | z_{k} = 1)}{\displaystyle \sum_{j=1}^{K} p(z_{j} = 1)p(\mathbf{x} | z_{j} = 1)} = \frac{\phi_{k}\mathcal{N}(\xbold | \mu_{k}, \Sigma)}{\displaystyle \sum_{j=1}^{K} \phi_{j}\mathcal{N}(\mathbf{x} | \mu_{j}, \Sigma)}
\end{equation}

where \(\mathcal{N}(\mathbf{x} | \mu_{i}, \Sigma) = \frac{1}{\sqrt{2\pi\alpha}}\exp\left(-\frac{||\xbold - \mu_{i}||^{2}}{2\alpha}\right) \)

Note that for the \(n^{th}\) datapoint \(\xbold_{n}\), the probability for being assigned to the \(k^{th}\) cluster in GMM, is given by: 
\begin{equation}
\label{posterior}
p(z_{k} = 1 | \xbold_{n}) = \frac{\phi_{k}\exp\left(-\frac{||\xbold_{n} - \mu_{k}||^{2}}{2\alpha}\right)}{\displaystyle \sum_{j=1}^{K} \phi_{j}\exp\left(-\frac{||\xbold_{n} - \mu_{j}||^{2}}{2\alpha}\right)}
\end{equation}

As \(\alpha\) tends to \(0\) in the \ref{posterior}, the terms in the summation in the denominator will go to 0 except for that term whose \(||\xbold_{n} - \mu_{j}||\) is smallest. Hence for such \(j\)'s, the value of \(p(z_{j} = 1 | \xbold_{n})\) will be 1, since the other terms in the denominator will have already gone to \(0\). This is a \(0-1\) probability setting of a datapoint belonging to a particular cluster and also the assignment of a datapoint to its closest mean.

Note that in the GMM, the maximization step (with update), updates the mean as:
\begin{equation}
\mu_{k} = \frac{1}{\displaystyle \sum_{n=1}^{N}p(z_{k} = 1 | \xbold_{n})} \displaystyle \sum_{n=1}^{N}p(z_{k} = 1 | \xbold_{n})\xbold_{n}
\end{equation}

Since the values of \(p(z_{k} = 1 | \xbold_{n})\) are either \(0\) or \(1\), the update step becomes: 
\begin{equation}
\mu_{k} = \frac{\displaystyle \sum_{n=1}^{N} \xbold_{n} \ind_{p(z_{k} = 1 | \xbold_{n}) = 1}}{\displaystyle \sum_{n=1}^{N} \ind_{p(z_{k} = 1 | \xbold_{n}) = 1}}
\end{equation} which is same update as K-Means.

From an objective function perspective, the objective function for K-Means is:
\begin{equation}
J = \sum_{n=1}^{N} \sum_{k=1}^{K} \ind_{\xbold_{n} \rightarrow k} ||\xbold_{n} - \mu_{k}||^{2}
\end{equation}

Here \(\ind_{\xbold_{n} \rightarrow k}\) denotes the event of \(\xbold_{n}\) being assigned to cluster \(k\). Similarly, the log likehihood for all training points in GMM for the limiting case (described above) is given by:
\begin{equation}
J' = -\frac{1}{2}\sum_{n=1}^{N} \sum_{k=1}^{K} p(z_{k} = 1 | \xbold_{n}) \ln(\mathcal{N}(\xbold_{n}|\mu_{k}, \Sigma)) = -\frac{1}{2}\sum_{n=1}^{N} \sum_{k=1}^{K} p(z_{k} = 1 | \xbold_{n}) ||\xbold_{n} - \mu_{k}||^{2} + C
\end{equation}

\(C\) is a constant term. For GMM, \(J'\) has to be maximized, and the \(J\) has to be minimized for K-Means. Note that since \(p(z_{k} = 1 | \xbold_{n})\) takes values \(0\) or \(1\), this is the same as \(\ind_{\xbold_{n} \rightarrow k}\). Also due to the negative sign in \(J'\), the maximization of \(J'\) is the same as minimization \(-J'\) which is the same as \(J\), excluding the constant \(C\).
\end{flushleft}
\section*{Question 3}

% TODO: Change the answer of Part a
\subsection*{Part a}
\begin{flushleft}
Below is the neural network that achieves the desired output i.e., the \texttt{XOR} function. 
\begin{figure}[H]
\centering
\includegraphics[width=0.4\linewidth]{./images/3a.png}
\end{figure}
Activations are present in both the hidden and output layers.
\end{flushleft}

\subsection*{Part b}
\begin{flushleft}
Below is the neural network that achieves the desired output i.e., the \texttt{XOR} function.
\begin{figure}[H]
\centering
\includegraphics[width=0.4\linewidth]{./images/3b.png}
\end{figure}
Activations are present in both the hidden and output layers.

For both Part a and Part b, there is a supporting script that will help you verify if the answers are correct or not. This is the file \texttt{checker.py}.
\end{flushleft}

\subsection*{Part c}
\subsubsection*{Part i}
Note that \(\sigma(z) = \tanh(z) = \frac{e^z - e^{-z}}{e^z + e^{-z}} = \frac{1 - e^{-2z}}{1 + e^{-2z}} = 2\times\text{sigmoid}(2z) - 1\). Using this fact we get:
\begin{gather}
\sigma'(z) = 2\times \text{sigmoid}'(2z) = 2\times 2\times \text{sigmoid}(2z)\times(1 - \text{sigmoid}(2z)) = 2\times(\tanh(z) + 1)\left(1 - \frac{\tanh(z) + 1}{2}\right) \\
\sigma'(z) = (1 - \tanh(z))(1 + \tanh(z)) = \boxed{(1 - \sigma(z))(1 + \sigma(z))}
\end{gather}
\subsubsection*{Part ii}
\begin{flushleft}
Note that node after the softmax applied for a layer is as follows, considering \(K\) nodes:
\begin{gather}
p_{i} = \frac{e^{z_{i}}}{e^{z_{i}} + N} \\
\frac{\partial p_{i}}{\partial z_{j}} = \begin{cases} \frac{Ne^{z_{i}}}{(e^{z_{i}} + N)^2} = (1 - p_{i})(p_{i}) & \text{ if i = j } \\ -\frac{e^{z_{i}}e^{z_{j}}}{(e^{z_{i}} + N)^2} = -p_{i}p_{j} & \text{ otherwise } \end{cases}
\end{gather}
where N is the rest of the normalization factor (written for convenience).

Using these two equations:
\begin{equation}
\frac{\partial L}{\partial z_{j}} = -\displaystyle \sum_{i=1}^{K} \frac{y_{i}}{p_{i}} \frac{\partial p_{i}}{\partial z_{j}} = -y_{j}(1 - p_{j}) + \sum_{i=1 , i \neq j}^{K} p_{j}y_{i} = -y_{j} + \sum_{i=1}^{K} p_{j}y_{i} = -y_{j} + p_{j}\sum_{i=1}^{K}y_{i} = \boxed{p_{j} - y_{j}}
\end{equation}

In the third equality, \(y_{i}p_{i} = y_{i}p_{j}\) if \(j = i\). This term is pushed inside the summation, for the rest of the equation.
\end{flushleft}

\section*{Question 4}
\subsection*{Part a}
\begin{flushleft}
\begin{equation}
J(\beta_{1}, \beta_{2}) = (Y - A\beta)^{T}W(Y - A\beta) = (Y^{T} - \beta^{T}A^{T})W(Y - A\beta) = Y^{T}WY - \beta^{T}A^{T}WY - Y^{T}WA\beta + \beta^{T}A^{T}WA\beta
\end{equation}

Using the ``trick" shown in Andrew Ng's CS229 at Stanford below, we get:
\begin{multline}
J(\beta_{1}, \beta_{2}) = \tr(J(\beta_{1}, \beta_{2})) = \tr(Y^{T}WY - \beta^{T}A^{T}WY - Y^{T}WA\beta + \beta^{T}AWA\beta) \\ = \tr(Y^{T}WY) - \tr(\beta^{T}A^{T}WY) - \tr(Y^{T}WA\beta) + \tr(\beta^{T}A^{T}WA\beta)
\end{multline}

% add link in the footnote below
The above is a ``trick" used in Andrew Ng's CS229 at Stanford. Using some properties of traces\footnote{CS229-Stanford-Lecture-Notes}:
\begin{itemize}
\item \(\tr(ABC) = \tr(BCA) = \tr(CAB)\)
\item \(\nabla_{A} \tr(AB) = \nabla_{A} \tr(B^TA^T) = B^{T}\)
\item \(\nabla_{A} \tr(ABA^{T}C) = CAB + C^{T}AB^{T}\)
\end{itemize}

We can hence write: 
\begin{itemize}
\item \(\tr(\beta^{T}A^{T}WY) = \tr(A^{T}WY\beta^{T}) \implies \nabla_{\beta} \tr(\beta^{T}A^{T}WY) = A^{T}WY\)
\item \(\tr(Y^{T}WA\beta) = \tr(\beta Y^{T}WA) \implies \nabla_{\beta} \tr(Y^{T}WA\beta) = A^{T}W^{T}Y = A^{T}WY\) (Since \(W\) is symmetric)
\item \(\tr(\beta^{T}A^{T}WA\beta) = \tr(\beta I \beta^{T}A^{T}WA) \implies \nabla_{\beta} \tr(\beta^{T}A^{T}WA\beta) = A^{T}WA\beta + A^{T}W^{T}A\beta = 2A^{T}WA\beta\) (Since \(W\) is symmetric)
\end{itemize}

Computing the gradient of \(J\) w.r.t. \(\beta\) and setting to \(\mathbf{0}\), we get:
\begin{gather}
\nabla_{\beta} J(\beta_{1}, \beta_{2}) = -2A^{T}WY + 2A^{T}WA\beta = \mathbf{0} \implies (A^{T}WA)\beta = A^{T}WY \\
\hat{\beta} = \boxed{(A^{T}WA)^{-1}(A^{T}WY)}
\end{gather}
\end{flushleft}

\subsection*{Part b}
The closed form solution in Part a will only exist if \(A^{T}WA\) is invertible. Since the inverse (if exists) is unique, this solution will be consequently unique as well. Otherwise there might exist multiple \(\hat{\beta}\)'s which satisfy the above equation.

\subsection*{Part c}
The update equation for gradient descent with learning rate = \(\eta\) is as follows:
\begin{equation}
\beta := \beta - \eta \nabla_{\beta}J(\beta_{1}, \beta_{2}) = (I - 2\eta A^{T}WA)\beta + 2\eta(A^{T}WY)
\end{equation}
\end{document}
